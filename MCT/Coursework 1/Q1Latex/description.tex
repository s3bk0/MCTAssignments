\documentclass[11pt,a4paper]{article}
\usepackage{xcolor}
%\usepackage[german]{babel}
\usepackage[T1]{fontenc}
\usepackage{amsmath}
\usepackage{amsfonts}
\usepackage{amssymb}
\usepackage{graphicx}
\usepackage{enumitem}
\usepackage{subcaption}%\begin{subfigure}{Breite der Subfigure}
\usepackage[left=3cm,right=3cm,top=2cm,bottom=2cm]{geometry}
\usepackage{physics}
\title{Coursework 1}
\date{}
%\author{Sebastian Kock}
\begin{document}
\maketitle
\newcommand{\num}{\thenum \stepcounter{num} }
\newcommand{\aufgabe}[1]{\subsection*{Aufgabe A\thesubsection: #1}
    \stepcounter{subsection}}
% \setcounter{section}{•}
\setcounter{subsection}{1}
\newcommand{\enumalph}{\renewcommand{\theenumi}{\alph{enumi}}}

\section*{Description Q1}
The task was to find numerical solutions to the differential
equation 
\begin{equation} \label{eq:DEQ}
    \dv[2]{w}{t} = (4g(2t)-s)\ w(t)
\end{equation}
with periodic boundary conditions $w(0) = w(2\pi)$. $g(t)$ is a
triangular periodic signal approximating $\cos(t)$ as shown in the 
assignment text.

To apply the method of finite differences the time interval $[0,2\pi]$
is discretised to a finite number $N$ of evenly spaced values 
$t_i,\ i\in I_N:=\{i\in \mathbb{N}_0\ | \ i\leq N\}$. The boundary values
are defined as $t_0=0$ and $t_N=2 \pi$.
Then, the second derivative of  $w(t)$ can be approximated by the 
\textit{three-point-difference} formula
\begin{equation}
    \dv[2]{w}{t} = \frac{w(t_{i+1})-2 w(t_i)+w(t_{i-1}) )}{\Delta t^2},
    \quad \Delta t := t_{i+1}-t_i\ i\in I_N
\end{equation}
known from the weekly material. Inserting this into equation \ref{eq:DEQ}
yields a system of N linear equations
\begin{equation}
    \frac{w(t_{i+1})-2 w(t_i)+w(t_{i-1}) }{\Delta t^2}
    =  (4g(2t_i)-s)\ w(t_i)\quad  \forall i\in I_N \setminus \{N\} \ .
\end{equation}
The boundary conditions can be expressed as $w(t_0)=w(t_{N})$
so that $w(t_{-1}) = w(t_{N-1})$ which is important for the equations
with $i=0$ and $i=N-1$ and enables to leave out $i=N$ in the calculation. 

In order to solve these equations they need to be expressed in matrix
vector form. Multiplying by $\Delta t^2$ and adding $4 \Delta t^2 g(2t_i)$ 
to each equation yields the expression
\begin{equation}
    \left( 2+4 \Delta t^2 g(2t_i) \right) w(t_i) - w(t_{i+1}) -w(t_{i-1}) 
    = \Delta t^2 s\ w(t_i) \forall i\in I_N \setminus \{N\} \ .
\end{equation}
Because of the boundary conditions, $w(t_N)$ does
not need to be calculated so that the equations with mentioned $i=0$ and $i=N-1$
are
\begin{align}
    \left( 2+4 \Delta t^2 g(2t_0) \right) w(t_0) - w(t_{1}) -w(t_{N-1}) 
    &= \Delta t^2 s\ w(t_0)  \\
    \left( 2+4 \Delta t^2 g(2t_{N-1}) \right) w(t_{N-1}) - w(t_0) -w(t_{N-2}) 
    &= \Delta t^2 s\ w(t_{N-1})
\end{align}
This can now be written in vector notation by constructing a
N dimensional vector 
\begin{equation}
    \vec{w}:=(w(t_0), w(t_1), ..., w(t_{N-1}))^T
\end{equation}
so that the equations become an eigenvalue problem
\begin{equation}
    \mathbf{A} \vec{w} = \lambda \vec{w} = \Delta t^2 s \vec{w}
\end{equation}
with $N\cross N$ matrix\footnote{see below for an example with $N=5$}
\begin{equation}
    \mathbf{A} = 
    \begin{pmatrix}
        d_0 & -1 & 0 & \cdots & 0 & -1 \\
        -1  & d_1   & -1    & 0 & \cdots & 0 \\
        0   &   -1  &   d_2 &   -1  & \cdots & 0 \\
        \vdots  & 0  & -1  & \ddots & 0 & 0 \\
        0   &   \vdots &  \vdots & 0 & \ddots & \vdots \\
        -1  &   0   &   0   &   0   & \cdots & d_{N-1}
    \end{pmatrix}, \quad d_i = 2+ 4 \Delta t^2 g(2t_i)\ .
\end{equation}
This problem can be easily implemented and solved by eigenvalue solvers.
The resulting eigenvalues $\lambda$ need to be divided by $\Delta t^2$
in order to obtain the $s$ of a solution. The calculated eigenvectors
represent periodic solutions and can be plotted over $t$.

\subsection*{Example with N=5}
For more clarity the case with $N=5$ is shown here
\begin{equation}
        \mathbf{A} = 
        \begin{pmatrix}
            d_0 & -1 & 0 & 0  & -1 \\
            -1  & d_1   & -1    & 0 & 0 \\
            0   &   -1  &   d_2 &   -1   & 0 \\
            0  & 0  & -1  & d_3 & -1  \\
            -1   & 0  &  0 & -1& d_4 \\
        \end{pmatrix}, \quad d_i = 2+ 4 \Delta t^2 g(2t_i)\ .
\end{equation}
\end{document}